%===== settings =====
%\expandafter\ifx\csname ifdraft\endcsname\relax
%	\documentclass[a4paper]{article}
%	\usepackage{bxcjkjatype}
%	\setlength{\textwidth}{480pt}
%	\setlength{\oddsidemargin}{-0.5cm}
%	\setlength{\evensidemargin}{-0.5cm}
%	\usepackage{amsmath}
%	\usepackage[unicode, bookmarks=true,colorlinks=true]{hyperref}
%	\setminchofont{ipaexm.ttf}
%	\setgothicfont{ipaexg.ttf}
%	\begin{document}
%\fi
%===== settings =====
\expandafter\ifx\csname ifdraft\endcsname\relax
	\documentclass[a4paper]{article}
	\setlength{\textwidth}{480pt}
	\setlength{\oddsidemargin}{-0.5cm}
	\setlength{\evensidemargin}{-0.5cm}
	\usepackage{amsmath}
	\usepackage{CJK}
	\begin{document}
	\begin{CJK*}{UTF8}{min}
\fi

%===== text =====
\section{序章}
\subsection{執筆にあたって}

%私は、設計システム技術センターの清家剛と申します。
%私は現在、入社3年目で、
%ずっと冷蔵庫の省エネ開発に関わらせて頂いています。

私は会社に入って伝熱工学を勉強しました。
私の専門は音響学で伝熱現象に対する理解はありませんでしたが、
専門分野の勉強は人並み以上にはやってきたつもりだったので、
伝熱工学も原理さえ押さえれば理解できるだろうと思っていました。
また、連続体の力学については理解しているつもりだったので、
波動方程式と比較して少々式が複雑になっても、
同じように考えることができるだろうと考えていました。

意気揚々と伝熱工学の勉強を開始し、
熱伝導の考え方については違和感なく受け入れることができました。
固体力学でいうフックの法則と同じように考えることができるためです。
しかし、対流熱伝達については、理解することができませんでした。
その理由は

\begin{itemize}
	\item 乱流の定義が定性的でよくわからない。
	\item 基礎に戻って学ぼうと流体力学の教科書で勉強しようとすると、乱流や統計的考察など、連続体力学とは馴染みが薄い内容になる。
	\item 剪断応力が無いにも関わらず渦なるものが発生する。
	\item 渦や乱流が伝熱に大きな影響がある。
\end{itemize}


\subsection{目的}
\begin{equation}
	\pi = 3.14
\end{equation}


%===== settings =====
\expandafter\ifx\csname ifdraft\endcsname\relax
	\end{CJK*}
	\end{document}
\fi

